%
% Proposal to The Unicode Consortium to include IEC-5007, -8, -9,
% -10, and IEEE 1621 "Stand-by" sumbols in Unicode.
%
% For information on this file please contact Joe Loughry at
% Tel. +1 303 221 4380 (time zone GMT minus 7 hours) or Email:
% joe.loughry@stx.ox.ac.uk
%

\documentclass[10pt,letterpaper]{article}

\usepackage[english,british]{babel}

% to let me use the new TrueType font we created:
\usepackage{fontspec}

% for formatting URLs (the `obeyspaces' option is for URLs with spaces (%20) in them):
\usepackage[obeyspaces]{url}

% added 20130901.2347 for putting angled brackets around URLs:
\newcommand{\URL}[1]{$\langle$\url{#1}$\rangle$}

% for XeLaTeX logo
\usepackage{hologo}

% Use \IEC{x} to display a symbol in the new font.
\newcommand{\IEC}[1]{{\fontspec{IECpower}#1}}

\begin{document}

\title{Proposal to Include IEC Power Button Symbols in Unicode}

\author{Joe Loughry and Terence Eden}

\maketitle

\begin{abstract}
The international symbols IEC 60417-5009 \IEC{P} meaning `power' and
IEEE-STD-1621 \IEC{S} meaning `stand-by' are not in Unicode.  Clearly
these signs would be useful to anyone writing technical or user manuals.
Furthermore, for electronically published documentation, it is crucial
to have the symbols defined in Unicode because it makes them searchable
in text.  In this proposal we provide a TrueType font named `IEC power'
containing the glyphs as specified in three international standards
together with all needed character properties.
\end{abstract}

\section{Introduction}

The following symbols are defined in IEC 60417, which is also ISO 7000:2012,
and IEEE 1621, which describes the current practice
\cite{IEEE1621,ISO7000,IEC60417}.

The five symbols defined in this font are shown in Table~\ref{table:symbols}.
Only these symbols are defined; if an undefined character, for example `A' is
called for in the font, the result is implementation-defined. In \hologo{XeTeX}
the result is \IEC{A}. In OpenOffice Writer, the result is `A' in a san-serif
typeface.

\begin{table}[htbp]
	\centering
	\begin{tabular}{clcll}
		\textbf{Symbol} & \textbf{Standard} & \textbf{Character} & \
			\textbf{Mnemonic} & \textbf{Meaning} \\
		\hline \\
		\IEC{P} & IEC 60417-5009 & P & `power'       & Power        \\
		\IEC{T} & IEC 60417-5010 & T & `toggle'      & Power on/off \\
		\IEC{0} & IEC 60417-5008 & 0 & `binary zero' & Power off    \\
		\IEC{1} & IEC 60417-5007 & 1 & `binary one'  & Power on     \\
		\IEC{S} & IEEE 6121      & S & `sleep'       & Stand-by     \\
    \end{tabular}
    \caption{All of the available glyphs in the \emph{IEC power} TrueType font.}
    \label{table:symbols} % label must come after caption!
\end{table}

\section{How this document was made}

There is an excellent SVG font tutorial at
\URL{http://www.webdesignerdepot.com/2012/01/how-to-make-your-own-icon-webfont/}
that is specifically aimed at generating icon fonts and includes an SVG font
starter file, instructions for using the SVG font editor in Inkscape,
recommendations about which on-line font converters are reliable, and tips for
editing the metadata and distributing the new font afterwards.

\subsection{TrueType fonts in {\LaTeX}}

To use the TrueType font in \TeX, this file must be compiled with \hologo{XeTeX}
and the font should be installed on the system ahead of time.

\bibliographystyle{plain}
\bibliography{consolidated_bibtex_file}

\vfill
{\tiny Build \input{build_counter.txt}}

\end{document}

